Packages are collections of R functions, data, and compiled code in a well-defined format. The directory where packages are stored is called the library. R comes with a standard set of packages. 

Others are available for download and installation.

\begin{framed}
\begin{verbatim}
library()   # see all packages installed 
search()    # see packages currently loaded
\end{verbatim}
\end{framed}

\subsection{Adding Packages}
You can expand the types of analyses you do be adding other packages. A complete list of contributed packages is available from CRAN.
Follow these steps:

Download and install a package (you only need to do this once).
To use the package, invoke the library(package) command to load it into the current session. 
(You need to do this once in each session, unless you customize your environment to automatically load it each time.)

On MS Windows:
Choose Install Packages from the Packages menu.
Select a CRAN Mirror. (e.g. Ireland)
Select a package. (e.g. chemCal)

Then use the library(package) function to load it for use. (e.g. library(chemCal))

You can install a package directly using the \texttt{install.packages()} command. Once installed you can directly call the library, using the ibrary command. 

Important - some packages require the more recent version of R to be installed.

\begin{framed}
\begin{verbatim}
install.packages("MethComp")
install.packages("chemCal")

library(MethComp)
library(chemCal)
\end{verbatim}
\end{framed}
