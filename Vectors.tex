

R Workshop
Introduction to R
Precision
Data Manipulation
data manipulation
Descriptive and Quantile Statistics
 
Vectors

Vectors are the simplest type of object in R. There are 3 main types of vectors:

  Numeric vectors
  Character vectors
  Logical vectors
  (Complex number vectors)

To set up a numeric vector x consisting of 5 numbers, 10.4, 5.6, 3.1, 6.4, 21.7, 

We use the c() command to concatenate them

> x = c(10.4, 5.6, 3.1, 6.4, 21.7)


To print the contents of x:

> x

[1] 10.4 5.6 3.1 6.4 21.7

The [1] in front of the result is the index of the first element in the vector x.

To access a particular element of x

> x[1]
[1] 10.4

> x[5]
[1] 21.7

We can also do further assignments:

> y <- c(x, 0, x)

This creates a vector y with 11 entries (two copies of x with a 0 in the middle)

 In R Computations are performed element-wise, e.g.
> 1/x

[1] 0.09615385 0.17857143 0.32258065 0.15625000 0.04608295



  Short vectors are "recycled" to match long ones

> v <- x + y

Warning message:

In x + y : longer object length is not a multiple of shorter object length


 
Some functions take vectors of values and produce results of the same length:
sin, cos, tan, asin, acos, atan, log, exp, ......

> cos(x)
[1] -0.5609843 0.7755659 -0.9991352 0.9931849 -0.9579148



 Some functions return a single value:
sum, mean, max, min, prod, ......

> sum(x)
[1] 47.2
>
> length(x)
[1] 5
>
> sum(x)/length(x)
[1] 9.44
>
> mean(x)
[1] 9.44

Some other interesting functions 
cumsum, sort, range, pmax, pmin, : : :

Generating Sequences 
R has a number of ways to generate sequences of numbers. 
These include the simplest approach, using the the colon ":", e.g.

> 1:10
[1] 1 2 3 4 5 6 7 8 9 10

This operator has the highest priority within an expression, e.g. 2*1:10 is equivalent to 2*(1:10).
 
The seq() function. (Use > ?seq to find out more about this function).

> seq(1,10)
> seq(from=1, to=10)
> seq(to=10, from=1)

are all equivalent to 1:10.


We can also specify a step size (using by=value) or a length (using length=value) for the sequence.

> s1 <- seq(1,10, by=0.5)
> s1
[1] 1.0 1.5 2.0 2.5 3.0 3.5 4.0 4.5 5.0 5.5 6.0
[12] 6.5 7.0 7.5 8.0 8.5 9.0 9.5 10.0
> s2 <- seq(1,10, length=5)
> s2
[1] 1.00 3.25 5.50 7.75 10.00
  
The rep() function replicates objects in various ways.

> s3 <- rep(x, 2)
>
> s3
[1] 10.4 5.6 3.1 6.4 21.7 10.4 5.6 3.1 6.4 21.7
[11] 10.4 5.6 3.1 6.4 21.7
>
> s4 <- rep(c(1,4),c(10,15))
> s4
[1] 1 1 1 1 1 1 1 1 1 1 4 4 4 4 4 4 4 4 4 4 4 4 4 4 4

 Character Vectors:

To set up a character/string vector z consisting of 4 place names use quotation marks.

> z <- c("Canberra", "Sydney", "Newcastle", "Darwin")

> z <- c(`Canberra', `Sydney', `Newcastle', `Darwin')

> c(z, "Mary", "John")

[1] "Canberra" "Sydney" "Newcastle" "Darwin" "Mary" "John"

There are a lot of in-built functions in R to manipulate character vectors.

Logical Vectors
 A logical vector is a vector whose elements are TRUE, FALSE or NA.
 These can be generated by logical conditions, e.g.

> temp <- x > 13

Takes each element of the vector x and compares it to 13.
Returns a vector the same length as x, with a value TRUE when the condition is met and FALSE when it is not.
 
The logical operators are >, >=, <, <=, == for exact equality and != for inequality.


Missing Values
In some cases the entire contents of a vector may not be known. For example, missing data from a particular data set.
A place can be reserved for this by assigning it the special value NA.

We can check for NA values in a vector x using the command
> is.na(x)

This returns a logical vector the same length as x with a value TRUE if that particular element is NA.

> w <- c(1:10, rep(NA,4), 22)
> is.na(w)

Indexing Vectors
We have already seen how to access single elements of a vector.

Subsets of a vector may also be selected using a similar approach.

  > ex1 <- w[!is.na(w)]
Stores the elements of the vector w that do NOT have the value NA, into ex1.

  > ex2 <- w[1:3]

Selects the fi rst 3 elements of the vector w and stores them in the new vector ex2.

  > ex3 <- w[-(1:4)]

Using the - sign indicates that these elements should be excluded. 
This command excludes the first 4 elements of w.

> ex4 <- w[-c(1,4)]

In this case only the 1st and 4th elements of w are excluded.

Modifying Vectors

To alter the contents of a vector, similar methods can be used.
 
Remember x has contents
> x
[1] 10.4 5.6 3.1 6.4 21.7

For example, to modify the 1st element of x and assign it a value 5 use
> x[1] <- 5
> x
[1] 5.0 5.6 3.1 6.4 21.7

Factors
A factor is a special type of vector used to represent categorical data, e.g. gender, social class, etc.

 Stored internally as a numeric vector with values 1, 2, ..... k, where k is the number of levels.
 Can have either ordered and unordered factors.
 A factor with k levels is stored internally consisting of 2 items
			(a) a vector of k integers
			(b) a character vector containing strings describing what the k levels are.



Vectors: Data Manipulation

Types of vectors
	logical
	numeric
	character
    Others  - Complex and Colour

creating a vector

scan()

X=c()

data.entry()

data manipulation
sort()
rev()    	Reverses the order of a vector
rep(,n)    Repeats a specified value or vector n times
length()   Returns the length of a vector
order()   Ordering of a vector

 
conversion and coercion

as.numeric()
as.integer()


Vectors: Indices

 

 


Descriptive and Quantile Statistics


Some simple ones to start with

mean()
range()
min()
max()
median()
IQR()
fivenum()


Measures of Dispersion
  range
  variance
  covariance
  standard deviation

sum
