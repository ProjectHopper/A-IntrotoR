
\documentclass{article}
\usepackage{natbib}
\usepackage{vmargin}
\usepackage{graphicx}
\usepackage{epsfig}
\usepackage{subfigure}
%\usepackage{amscd}
\usepackage{amssymb}
\usepackage{amsbsy}
\usepackage{amsthm, amsmath}
%\usepackage[dvips]{graphicx}
\bibliographystyle{chicago}
\renewcommand{\baselinestretch}{1.0}

\begin{document}

\section*{Exercise 25: Confidence intervals}
Seven meas urements of the pH of a buffer solution gave the following results:
5.12 5.20 5.15 5.17 5.16 5.19 5.15. Calculate (i) the 95\% and (ii) the 99\% confidence limits for the true pH
utilizing R.

\begin{verbatim}
> x<-c(5.12, 5.20, 5.15, 5.17, 5.16, 5.19, 5.15)
> n=length(x)
>
> alpha5=0.05
> alpha1=0.01
>
> LB95=mean(x)+qt(alpha5/2,n-1)*sd(x)/sqrt(n)
> UB95=mean(x)+qt(1-alpha5/2,n-1)*sd(x)/sqrt(n)
>
> LB95
[1] 5.137975
> UB95
[1] 5.187739

\end{verbatim}

Thus the 95\% confidence interval for this problem is approximately [5:14; 5:19], while
using the same code only taking $\alpha$ =0.01 will give us [5:13; 5:20].

\newpage
\section*{Exercise 26: Correlation test}

Assessment of tuna quality: We compare the Hunter L measure of
lightness to the averages of consumer panel scores (recoded as
integer values from 1 to 6 and averaged over 80 such values) in
 9 lots of canned tuna.
(Hollander \& Wolfe (1973), p. 187f.)

\begin{verbatim}
x <- c(44.4, 45.9, 41.9, 53.3, 44.7, 44.1, 50.7, 45.2, 60.1)
y <- c( 2.6,  3.1,  2.5,  5.0,  3.6,  4.0,  5.2,  2.8,  3.8)
\end{verbatim}

\begin{itemize}

\item Test the hypothesis that the correlation coefficient is not zero.
\item Test the hypothesis that the correlation coefficient is positive.
\item What is the test statistics in both cases?
\item What is the p-value in both cases?
\item Interpret the p-values.
\end{itemize}


The alternative hypothesis of interest is there is a correlation between the
Hunter L value and the panel score.

\begin{verbatim}
> cor.test(x, y , alternative = "two.sided")

        Pearson's product-moment correlation

data:  x and y
t = 1.8411, df = 7, p-value = 0.1082
alternative hypothesis: true correlation is not equal to 0
95 percent confidence interval:
 -0.1497426  0.8955795
sample estimates:
      cor
0.5711816

\end{verbatim}


The alternative hypothesis of interest is that the
Hunter L value is positively associated with the panel score.
\begin{verbatim}
> cor.test(x, y, alternative = "greater")

        Pearson's product-moment correlation

data:  x and y
t = 1.8411, df = 7, p-value = 0.05409
alternative hypothesis: true correlation is greater than 0
95 percent confidence interval:
 -0.02223023  1.00000000
sample estimates:
      cor
0.5711816
\end{verbatim}

\begin{verbatim}
> ct2 <- cor.test(x, y, alternative = "greater")
> names(ct2)
[1] "statistic"   "parameter"   "p.value"   "estimate"  "null.value"  "alternative" "method"      "data.name"
[9] "conf.int"
>
> ct2$p.value
[1] 0.05408653
>
> ct2$statistic
       t
1.841083
\end{verbatim}

\section*{Exercise 27: Kolmogorov-Smirnov test}

\begin{verbatim}
x <- rnorm(50)
y <- runif(30)
# Do x and y come from the same distribution?
ks.test(x, y)

        Two-sample Kolmogorov-Smirnov test

data:  x and y
D = 0.46, p-value = 0.0004387
alternative hypothesis: two-sided
\end{verbatim}


\section*{Exercise 28: Shapiro-Wilk test}

\begin{verbatim}
> x<-rnorm(100, mean = 5, sd = 3)
> y<-runif(100, min = 2, max = 4)
> shapiro.test(x)

        Shapiro-Wilk normality test
data:  x
W = 0.9851, p-value = 0.3249
>
> shapiro.test(y)
        Shapiro-Wilk normality test
data:  y
W = 0.9585, p-value = 0.003151

\end{verbatim}
\newpage


\section*{Exercise 29}

Add up all the numbers for 1 to 50
\begin{itemize}
\item  Using the sum() command.
\item  using a ‘for’ loop.
\end{itemize}
\newpage








%---------------------------------------------------------------------------------------------%
\newpage



\begin{itemize}
\item Exercise 14: question on trigonometric functions omitted.
\item Exercise 17: inverse cosine function is acos()
\begin{verbatim}
> cos(pi)
[1] -1
> acos(-1)
[1] 3.141593
>
\end{verbatim}
\item Exercise 18: use the standard normal distribution to generate random numbers, unless told otherwise.

\begin{verbatim}
> rnorm(4)
[1] -1.68732684 -0.62743621  0.01831663  0.70524346
\end{verbatim}
\item Exercise 18: We have not covered the material for parts 5, 10 and 11 yet.
\end{itemize}

\section*{Exercise 20}

Construct a vector of the 0.9, 0.95, 0.975 and 0.99 quantiles of the $t-$ distribution (degrees of freedom = 16).
For hep with the $t-$ distribution use the command '?qt'.

\begin{verbatim}
> perc =c(0.9, 0.95, 0.975, 0.99)
> perc
[1] 0.900 0.950 0.975 0.990
> qt(perc, 16)
[1] 1.336757 1.745884 2.119905 2.583487
\end{verbatim}

What is the 0.975 quantile for the $t-$ distribution when the degrees of freedom is 40?
(N.B. This exercise is relevant to forthcoming topics, such as confidence intervals.)

\begin{verbatim}

> qt(0.975, 40)
[1] 1.336757 1.745884 2.119905 2.583487
\end{verbatim}
\section*{Exercise 21}

\begin{table}[ht]
\begin{center}
\begin{tabular}{rlllrrl}
  \hline
 & Make & Model & Cylinder & Weight & Mileage & Type \\
  \hline
1 & Honda & Civic & V4 & 2170.00 & 33.00 & Sporty \\
  2 & Chevrolet & Beretta & V4 & 2655.00 & 26.00 & Compact \\
  3 & Ford & Escort & V4 & 2345.00 & 33.00 & Small \\
  4 & Eagle & Summit & V4 & 2560.00 & 33.00 & Small \\
  5 & Volkswagen & Jetta & V4 & 2330.00 & 26.00 & Small \\
  6 & Buick & Le Sabre & V6 & 3325.00 & 23.00 & Large \\
  7 & Mitsbusihi & Galant & V4 & 2745.00 & 25.00 & Compact \\
  8 & Dodge & Grand Caravan & V6 & 3735.00 & 18.00 & Van \\
  9 & Chrysler & New Yorker & V6 & 3450.00 & 22.00 & Medium \\
  10 & Acura & Legend & V6 & 3265.00 & 20.00 & Medium \\
   \hline
\end{tabular}
\end{center}
\end{table}



1) Create a data frame for the data ( see `Car.R')
\begin{verbatim}
> Car<-data.frame(Make,Model,Cylinder,Weight,Mileage,Type)
\end{verbatim}

2) Find the `names' and the dimensions of the Car data frame.
\begin{verbatim}
> names(Car)
[1] "Make" "Model" "Cylinder" "Weight" "Mileage" "Type"
> dim(Car)
[1] 10  6
\end{verbatim}

3) Find the mean and standard deviation of the weight of the cars.
\begin{verbatim}
> mean(Car$Weight)
[1] 2858
> sd(Car$Weight)
[1] 543.7943
\end{verbatim}

4) Subset the data so that only cars above 2500 kgs are considered. (Call the subset 'cars2500')
\begin{verbatim}
cars2500 = Car[Car$Weight>2500,]
\end{verbatim}

5) Make a subset of the small cars (call the subset `smallcar').
\begin{verbatim}
> smallcar=Car[Car$Type=="Small",]
> smallcar
        Make  Model Cylinder Weight Mileage  Type
3       Ford Escort       V4   2345      33 Small
4      Eagle Summit       V4   2560      33 Small
5 Volkswagen  Jetta       V4   2330      26 Small
>
\end{verbatim}

6) What is the correlation coefficient between weight and mileage?

\begin{verbatim}
> cor(Car$Weight,Car$Mileage)
[1] -0.8759577
\end{verbatim}

7) Make a plot of weights against mileage (include a title).

\begin{verbatim}
>plot(Car$Weight, Car$Mileage, main="Weight vs. Mileage")
\end{verbatim}

Alternatively
\begin{verbatim}
>attach(Car)
>plot(Weight, Mileage, main="Weight vs. Mileage")
\end{verbatim}

\newpage


\newpage

\begin{verbatim}
# Car.R
> Make<-c("Honda","Chevrolet","Ford","Eagle","Volkswagen",
+"Buick","Mitsbusihi", "Dodge","Chrysler","Acura")
>
> Model<-c("Civic","Beretta","Escort","Summit","Jetta",
+"Le Sabre","Galant", "Grand Caravan","New Yorker","Legend")
>
> rep("V6",3)               # repeat three times
[1] "V6" "V6" "V6"
> Cylinder<-c(rep("V4",5),"V6","V4",rep("V6",3))
> Cylinder
[1] "V4" "V4" "V4" "V4" "V4" "V6" "V4" "V6" "V6" "V6"
> Weight<-c(2170,2655,2345,2560,2330,3325,2745,3735,3450,3265)
> Mileage<-c(33,26,33,33,26,23,25,18,22,20)
> Type<-c("Sporty","Compact",rep("Small",3),"Large",
+"Compact","Van",rep("Medium",2))
\end{verbatim}

\end{document}
